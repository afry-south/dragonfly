\documentclass[a4paper]{article}

\usepackage{url}
\usepackage{amsmath}
\usepackage{verbatim}   		% Useful for program listings
\usepackage[T1]{fontenc}       	% For Swedish characters ÅÄÖ etc.
\usepackage[utf8]{inputenc}
% \usepackage[swedish]{babel} % For Swedish hyphenation
\usepackage{fancyvrb}           	% For lists with tabulators
\fvset{tabsize=4}              	 	% Tabulator size
\fvset{fontsize=\small}         	% List font size
\usepackage{graphicx}		% Imports the graphicx package, useful for images

% generate clickable references & toc
\usepackage[]{hyperref}
\hypersetup{
%    pdftitle={Your title here},
%    pdfauthor={Your name here},
%    pdfsubject={Your subject here},
%    pdfkeywords={keyword1, keyword2},
    bookmarksnumbered=true,
    bookmarksopen=true,
    bookmarksopenlevel=1,
    colorlinks=true,
    pdfstartview=Fit,
    pdfpagemode=UseOutlines,    % this is the option you were lookin for
    pdfpagelayout=TwoPageRight
}

\title{ Use cases for\\ Dragonfly Project}
\author{Eduardo Riffo}
\date{\today}         		% Today's date if not specified

\begin{document}                	% Start of document

\maketitle                      	% Prints the title defined above with \title, \author and \date

\begin{center}
\vspace{64pt}
\includegraphics[scale=1.6]{images/AF_Logotype20141_Black.png}
\vspace{16pt}
\\ \large ÅF Embedded Systems
\end{center}

\vspace{16pt}
\begin{tabular}{ l l l p{8.5cm} }
	Ver. & Date & Name & Description and Reason for change \\\hline
	0.1 Draft & December 8, 2015 & Eduardo & Created document structure.\\
	0.2  & December 8, 2015 & Eduardo & Added usecase "Charge the quadcopter battery".\\
	0.3  & December 8, 2015 & Eduardo & Modified usecase "Charge the quadcopter battery".\\
	0.4	& December 9, 2015	& Nina	& Added usecase "Safety".\\
	0.5	& December 9, 2015	& Nina	& Modified usecase "Safety", including changing the name to "Safety of humans, objects and the quadcopter".\\
\end{tabular}

\newpage

\tableofcontents					% Insert table of contents

\newpage

\section{Introduction}

The \emph{Dragonfly} project is an internal competence enhancement project for ÅF employees. The goal is to combine technology, competence and experience from various engineering fields in order to construct a highly advanced quadrotor UAV system. This documents contains all current identified use cases for this project.

\section{Definitions and Abriviations}

\subsection{Actor}
An actor is a person or other entity external to the software system being specified who interacts with the system and performs use cases to accomplish tasks. Different actors often correspond to different user classes, or roles, identified from the customer community that will use the product. Name the actor(s) that will be performing this use case.
\subsection{Description}
Provide a brief description of the reason for and outcome of this use case, or a high-level description of the sequence of actions and the outcome of executing the use case.
\subsection{Preconditions}
List any activities that must take place, or any conditions that must be true, before the use case can be started. Number each precondition. Examples:
\begin{enumerate}
	\item User’s identity has been authenticated.
	\item User’s computer has sufficient free memory available to launch task.
\end{enumerate}
\subsection{Postconditions}
Describe the state of the system at the conclusion of the use case execution. Number each postcondition. Examples:
\begin{enumerate}
	\item Document contains only valid SGML tags.
	\item Price of item in database has been updated with new value.
\end{enumerate}
\subsection{Priority}
Indicate the relative priority of implementing the functionality required to allow this use case to be executed. The priority scheme used must be the same as that used in the software requirements specification.
\subsection{Priority}
Indicate the relative priority of implementing the functionality required to allow this use case to be executed. The priority scheme used must be the same as that used in the software requirements specification.
\subsection{Frequency of Use}
Estimate the number of times this use case will be performed by the actors per some appropriate unit of time.
\subsection{Normal Course of Events}
Provide a detailed description of the user actions and system responses that will take place during execution of the use case under normal, expected conditions. This dialog sequence will ultimately lead to accomplishing the goal stated in the use case name and description. This description may be written as an answer to the hypothetical question, “How do I <accomplish the task stated in the use case name>?” This is best done as a numbered list of actions performed by the actor, alternating with responses provided by the system
\subsection{Alternative Courses}
Document other, legitimate usage scenarios that can take place within this use case separately in this section. State the alternative course, and describe any differences in the sequence of steps that take place. Number each alternative course using the Use Case ID as a prefix, followed by “AC” to indicate “Alternative Course”. Example:  X.Y.AC.1.
\subsection{Exceptions}
Describe any anticipated error conditions that could occur during execution of the use case, and define how the system is to respond to those conditions. Also, describe how the system is to respond if the use case execution fails for some unanticipated reason. Number each exception using the Use Case ID as a prefix, followed by “EX” to indicate “Exception”. Example:  X.Y.EX.1.
\subsection{Includes}
List any other use cases that are included (“called”) by this use case. Common functionality that appears in multiple use cases can be split out into a separate use case that is included by the ones that need that common functionality.
\subsection{Special Requirements}
Identify any additional requirements, such as nonfunctional requirements, for the use case that may need to be addressed during design or implementation. These may include performance requirements or other quality attributes
\subsection{Assumptions}
List any assumptions that were made in the analysis that led to accepting this use case into the product description and writing the use case description.
\subsection{Notes and Issues}
List any additional comments about this use case or any remaining open issues or TBDs (To Be Determineds) that must be resolved. Identify who will resolve each issue, the due date, and what the resolution ultimately is.

\newpage
%%%%%%%%%%%%%%%%%%%%%%%%%%%%%%%%%%%%%%%%%
%				CHARGE
%%%%%%%%%%%%%%%%%%%%%%%%%%%%%%%%%%%%%%%%%
\section{Use Case: Charge the quadcopter battery}
\subsection{Use case details}

\begin{tabular}{lll}
	&Name  &Charge the quadcopter battery  \\
	&ID  &4380aef6-e6a3-46ad-8347-87973d381a4c9 ( UUID Generated by https://www.uuidgenerator.net/)  \\
	&Creation date  & December 8, 2015  \\
	&Created by  &Eduardo Riffo and Abed Shoka   
\end{tabular}

\subsection{Actor}
Quadcopter unit, Quadcopter user
\subsection{Description}
This use case covers the functionality provided by subsystem handling of the interaction between quadcopter and battery charging base. Use case includes functional and non-functional behaviors of the charging subsystem/classifier.

\subsection{Preconditions}
The following constraints must be true to be able to run the scenarios within this use case.
\begin{itemize}
	\item The actor must have receiver capability from a hardware perspective.
	\item The subsystem have transmitter capabilities from a hardware perspective.
	\item Ensure that the transmitter fulfills all the safety requirements.
\end{itemize}
\subsection{Postconditions}
The following constraint must be true after completing the scenarios of this use case.
\begin{itemize}
	\item Charging indicator on the receiver must have an improved battery value.
	\item Make sure that the transmitter do not overheat.
\end{itemize}
\subsection{Priority}
The scenarios of this use case, or rather it's subsystem, should have high priority to be able meet the classifiers objectives.
\subsection{Frequency of use}
The scenarios of this use case will occur frequently, several times a day depending on the actor usage.
\subsection{Normal course of events}
This use case has the following scenarios with the normal course of events:
\begin{itemize}
	\item Quadcopter unit initiated charging. From now on called \textit{automatic charging}.
	\item Quadcopter user initiated charging. From now on called \textit{manual charging}.
\end{itemize}
\subsection{Alternative courses}
If the charging subsystem fails to cope with either automatic or manual charging,
The following alternative scenarios must be followed:
\begin{itemize}
	\item Manual: Suggest alternative battery charging base.
	\item Automatic: Redirect to alternative battery charging base.
\end{itemize}
If the there is no alternative charging base or if alternative base is to far away,
the following alternative scenarios must be followed:
\begin{itemize}
	\item Manual: Inform quadcopter user that charging failed. Let user take action.
	\item Automatic: Close down quadcopter in an orderly manner.
\end{itemize}
\subsection{Exceptions}
If the charging subsystem fails to response due to exceptions, the following scenarios must be followed:
\begin{itemize}
	\item Manual: Inform quadcopter user that charging failed. Let user take action.
	\item Automatic: Close down quadcopter in an orderly manner.
\end{itemize}
\subsection{includes}
Safety UC, Connection UC, Navigation UC
\subsection{Special requirements}
Safety regulations regarding the receiver and transmitter must be very strict, so this must be measurable even if it is  a nonfunctional requirements. Voltage rails must be defined.
\subsection{Assumptions}
N/A
\subsection{Notes and issues}
Work ongoing to define the voltage budget used by the different hardware modules and thereby optimize charging form a hardware perspective.

%%%%%%%%%%%%%%%%%%%%%%%%%%%%%%%%%%%%%%%%%
%				SAFETY
%%%%%%%%%%%%%%%%%%%%%%%%%%%%%%%%%%%%%%%%%
\section{Use Case: Safety of humans, objects and the quadcopter}
\subsection{Use case details}

\begin{tabular}{lll}
	&Name  & Safety of humans, objects and the quadcopter\\
	&ID  & 8c77715a-d70d-4cf0-8e9c-5e18d3d2f2ca (UUID Generated by https://www.uuidgenerator.net/)  \\
	&Creation date  & December 9, 2015\\
	&Created by  & Nina Khayyami
\end{tabular}

\subsection{Actor}
Quadcopter unit, Quadcopter user
\subsection{Description}
This use case covers the safety of humans, objects and the quadcopter itself while the quadcopter is in use. The use case includes functional and non-functional behaviours of the safety aspect. 
\subsection{Preconditions}
The following constraints must be true to be able to run the scenarios within this use case.
\begin{itemize}
\item The quadcopter unit has parts that can harm or injure humans, objects or the quadcopter itself.
\item The quadcopter unit must be able to fly.
\end{itemize}
\subsection{Postconditions}
The following constraints must be true after completing the scenarios of this use case.
\begin{itemize}
\item Humans, objects or the quadcopter itself must not be harmed. The quadcopter unit may be harmed if it is the only way to avoid harming humans and/or other objects.
\end{itemize}
\subsection{Priority}
TBD
\subsection{Frequency of use}
The scenarios of this use case will occur frequently, depending on the actor usage.
\subsection{Normal course of events}
This use case has the following scenarios with the normal course of events:
\begin{itemize}
\item Quadcopter unit detects a human and/or an object and changes the route to avoid flying within the safety distance of the human and/or object.
\end{itemize}
\subsection{Alternative courses}
If the safety subsystem fails to avoid humans and/or objects, the following alternative scenarios must be followed:
\begin{itemize}
\item Quadcopter unit flies within the safety distance of a human and/or an object and automatically switches to panic-mode.
\end{itemize}
\subsection{Exceptions}
If the safety subsystem fails to response due to exceptions, the following scenarios must be followed:
\begin{itemize}
\item If the quadcopter unit flies within the safety distance of a human and/or and object and fails to automatically switch to panic-mode, the user must be able to switch to panic-mode through the RC controller.
\end{itemize}
\subsection{includes}
TBD
\subsection{Special requirements}
TBD
\subsection{Assumptions}
TBD
\subsection{Notes and issues}
TBD

%%%%%%%%%%%%%%%%%%%%%%%%%%%%%%%%%%%%%%%%%
%				TEMPLATE
%%%%%%%%%%%%%%%%%%%%%%%%%%%%%%%%%%%%%%%%%
\section{Use Case XXX Template}
\subsection{Use case details}

\begin{tabular}{lll}
	&Name  & TBD \\
	&ID  & TBD ( UUID Generated by https://www.uuidgenerator.net/) \\
	&Creation date  & TBD \\
	&Created by  & TBD 
\end{tabular}

\subsection{Actor}
TBD
\subsection{Description}
TBD
\subsection{Preconditions}
TBD
\subsection{Postconditions}
TBD
\subsection{Priority}
TBD
\subsection{Frequency of use}
TBD
\subsection{Normal course of events}
TBD
\subsection{Alternative courses}
TBD
\subsection{Exceptions}
TBD
\subsection{includes}
TBD
\subsection{Special requirements}
TBD
\subsection{Assumptions}
TBD
\subsection{Notes and issues}
TBD

\end{document}                  % End of document
