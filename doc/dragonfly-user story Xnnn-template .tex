\documentclass[a4paper]{article}

\usepackage{url}
\usepackage{amsmath}
\usepackage{verbatim}   	   	    % Useful for program listings
\usepackage[T1]{fontenc}       	    % For Swedish characters ÅÄÖ etc.
\usepackage[utf8]{inputenc}
% \usepackage[swedish]{babel}       % For Swedish hyphenation
\usepackage{fancyvrb}           	% For lists with tabulators
\usepackage{titlesec}               % Define for title sections
\fvset{tabsize=4}              	 	% Tabulator size
\fvset{fontsize=\small}         	% List font size
\usepackage{graphicx}		        % Imports the graphicx package, useful for images
\usepackage[parfill]{parskip}

% generate clickable references & toc
\usepackage[]{hyperref}
\hypersetup{
%    pdftitle={Your title here},
%    pdfauthor={Your name here},
%    pdfsubject={Your subject here},
%    pdfkeywords={keyword1, keyword2},
    bookmarksnumbered=true,
    bookmarksopen=true,
    bookmarksopenlevel=1,
    colorlinks=true,
    pdfstartview=Fit,
    pdfpagemode=UseOutlines,    % this is the option you were lookin for
    pdfpagelayout=TwoPageRight
}

\setcounter{secnumdepth}{5}


\title{Dragonfly Project \\ User Story XXnnnn}
\author{Eduardo Riffo}

\date{December, 2015}         		% Today's date if not specified



\begin{document}                	% Start of document

\maketitle                      		% Prints the title defined above with \title, \author and \date

\begin{center}
\vspace{64pt}
\includegraphics[scale=1.6]{images/AF_Logotype20141_Black.png}
\vspace{16pt}
\\ \large ÅF Embedded Systems
\end{center}

\newpage

\tableofcontents				% Insert table of contents

\newpage

\section{Introduction}

In an agile framework, user stories are the smallest units of work. The goal of a user story is to deliver a particular value back to the project.User stories are a few sentences in simple language that outline the desired outcome. They don't go into detailed requirements.

\section{User story - Structure}
	\subsection{User Story - ID}
	Format is : XXnnnn
	The first two characters indicates the subsystem and the rest numbering. This numbering is defined in the backlog document.
	Example : CH9999
	\subsection{Main objective}
	As a <type of user>, I want <goal> so that I <receive benefit>.

	Example:
	As a \textit{developer}, I want \textit{to implement a camera solution} so that I \textit{can monitor potential threats}.

	\subsection{Type of user}
	Type of user can be e.g Customer, project manager, Developer, tester, technical coordinator.
	Type of user can also be a group.

	\subsection{Goal}
	Must be a measurable goal.

	\subsection{receive benefit}
	List the benefit if goal is achived.

		
	\subsection{Software analysis and design}
	Describe the analysis outcome to describe the solution both from an analysis and design point of view.
	 
	\subsubsection{Analysis Models}
	List the analysis models used to meet the funtional requirements are coverered by this user story.  
		
	\paragraph{Sequence Diagrams}\mbox{} \\
	List all sequence diagrams needed for this user story

	\paragraph{Data Flow Diagrams - DFD}\mbox{} \\
	List all flow diagrams needed for this user story

	\paragraph{State-Transition Diagrams - STD}\mbox{} \\
	List all flow diagrams needed for this user story
	
	\subsubsection{Classes / Objects}
	List the classes and object that need to be created to be able to meet the goal.
	 
	\paragraph{Functions}\mbox{} \\
    List all new and/or modified functions for this user story.
    
	\paragraph{Attributes}\mbox{} \\
	List all new and/or modified attributes for this user story.

	
\section{Additional information}

	\subsection{Covered SRS items}
    List all the SRS Requirements that is tied to this story, may be one or many.
	
	\subsection{Affected documents}
    List all the Implementation Design Specification and Test Case Descrition specifications.
    These may be one or many-
	
	\subsection{Checked PRS items}
	List all PRS items that will be delivered by this user story.

\end{document}                  % End of document
