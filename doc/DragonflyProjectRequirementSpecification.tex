\documentclass[a4paper]{article}

\usepackage{url}
\usepackage{amsmath}
\usepackage{verbatim}   		% Useful for program listings
\usepackage[T1]{fontenc}       	% For Swedish characters ÅÄÖ etc.
\usepackage[utf8]{inputenc}
% \usepackage[swedish]{babel} % For Swedish hyphenation
\usepackage{fancyvrb}           	% For lists with tabulators
\fvset{tabsize=4}              	 	% Tabulator size
\fvset{fontsize=\small}         	% List font size
\usepackage{graphicx}		% Imports the graphicx package, useful for images

% generate clickable references & toc
\usepackage[]{hyperref}
\hypersetup{
%    pdftitle={Your title here},
%    pdfauthor={Your name here},
%    pdfsubject={Your subject here},
%    pdfkeywords={keyword1, keyword2},
    bookmarksnumbered=true,
    bookmarksopen=true,
    bookmarksopenlevel=1,
    colorlinks=true,
    pdfstartview=Fit,
    pdfpagemode=UseOutlines,    % this is the option you were lookin for
    pdfpagelayout=TwoPageRight
}

% define namedlabel as per http://texblog.org/2012/03/21/cross-referencing-list-items/
% this is used for naming the requirements.
\usepackage{enumitem, hyperref}
\usepackage{nameref}

% suggestion from http://tex.stackexchange.com/questions/1230/reference-name-of-description-list-item-in-latex
\makeatletter
\newcommand{\labitem}[2]{%
\def\@itemlabel{\textbf{#2}}
\item
\def\@currentlabel{#2}\label{#1}}
\makeatother


\title{Dragonfly Quadrotor UAV \\ Project Requirements Specification}
\author{Daniel Stenberg \\ Nina Khayyami \\ Daniel Nilsson \\ Eduardo Riffo \\ Adam Steineck}

\date{December, 2015}         		% Today's date if not specified


\begin{document}                	% Start of document

\maketitle                      		% Prints the title defined above with \title, \author and \date

\begin{center}
\vspace{64pt}
\includegraphics[scale=1.6]{images/AF_Logotype20141_Black.png}
\vspace{16pt}
\\ \large ÅF Technology
\end{center}

\newpage

\tableofcontents				% Insert table of contents

\newpage

\section{Introduction}

The \emph{Dragonfly} project is an internal competence enhancement project for ÅF Technology employees. The goal is to combine technology, competence and experience from various engineering fields in order to construct a highly advanced quadrotor UAV system. The focus of the Flight Control Board development deals with low-level maneuvering of the aircraft, calculating motor command based on a feedback control system. Some of the major technologies deployed to attain this are control theory, electronics and software development.

\section{Use cases}

\subsection{Use case 1: Navigate}
\label{uc:navigate}
\begin{enumerate}
	\labitem{req:nav.1}{Navigate.1}: Navigate to a destination
	\labitem{req:nav.2}{Navigate.2}: Navigate to start position
\end{enumerate}


\subsection{Use case 2: Connection}

\setlength{\parindent}{3cm}

\begin{enumerate}
\labitem{req:conn.1}{Connection.1}: The device shuold have a functional WiFi connection.
\labitem{req:conn.2}{Connection.2}: The device should have a functional internet connection.
\end{enumerate}

\section{Example: Referring to requirements}
Let's consider requirement \ref{req:nav.1} and \ref{req:nav.2}. Then consider \ref{req:conn.1}.


\begin{thebibliography}{99}
\bibitem{stenberg} Model-based Design Development and Control of a Wind Resistant Multirotor UAV, C. Månsson, D. Stenberg, Lunds Tekniska Högskola 2014
\end{thebibliography}

\end{document}                  % End of document
