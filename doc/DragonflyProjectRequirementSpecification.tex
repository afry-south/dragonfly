
\documentclass[a4paper]{article}

\usepackage{url}
\usepackage{amsmath}
\usepackage{verbatim}   		% Useful for program listings
\usepackage[T1]{fontenc}       	% For Swedish characters ÅÄÖ etc.
\usepackage[utf8]{inputenc}
% \usepackage[swedish]{babel} % For Swedish hyphenation
\usepackage{fancyvrb}           	% For lists with tabulators
\fvset{tabsize=4}              	 	% Tabulator size
\fvset{fontsize=\small}         	% List font size
\usepackage{graphicx}		% Imports the graphicx package, useful for images

% generate clickable references & toc
\usepackage[]{hyperref}
\hypersetup{
%    pdftitle={Your title here},
%    pdfauthor={Your name here},
%    pdfsubject={Your subject here},
%    pdfkeywords={keyword1, keyword2},
    bookmarksnumbered=true,
    bookmarksopen=true,
    bookmarksopenlevel=1,
    colorlinks=true,
    pdfstartview=Fit,
    pdfpagemode=UseOutlines,    % this is the option you were lookin for
    pdfpagelayout=TwoPageRight
}

% define namedlabel as per http://texblog.org/2012/03/21/cross-referencing-list-items/
% this is used for naming the requirements.
\usepackage{enumitem, hyperref}
\usepackage{nameref}

% suggestion from http://tex.stackexchange.com/questions/1230/reference-name-of-description-list-item-in-latex
\makeatletter
\newcommand{\labitem}[2]{%
\def\@itemlabel{\textbf{#2}}
\item
\def\@currentlabel{#2}\label{#1}}
\makeatother

% just a work-in-progress idea

% usage: \usecase{cross-reference}{display-reference} text text ...
\newcommand{\usecase}[2]{
  \subsection*{#2}
  \label{#1}
  \addcontentsline{toc}{subsection}{\nameref{#1}}
}


% usage: \scenario{cross-reference}{display-reference} text text ...
\newcommand{\scenario}[2]{
  \subsubsection*{#2}
  \label{#1}
  \addcontentsline{toc}{subsubsection}{\nameref{#1}}
}

% usage: \requirement{cross-reference}{display-reference} text text ...
\newcommand{\requirement}[2]{
  \paragraph*{#2:}
  \label{#1}
  \addcontentsline{toc}{paragraph}{\nameref{#1}}
}




\title{Dragonfly Quadrotor UAV \\ Project Requirements Specification}
\author{Daniel Stenberg \\ Nina Khayyami \\ Daniel Nilsson \\ Eduardo Riffo \\ Adam Steineck}

\date{\today}
%\date{December, 2015}         		% Today's date if not specified


\begin{document}                	% Start of document

\maketitle                      		% Prints the title defined above with \title, \author and \date

\begin{center}
\vspace{64pt}
\includegraphics[scale=1.6]{images/AF_Logotype20141_Black.png}
\vspace{16pt}
\\ \large ÅF Technology
\end{center}

\newpage

\tableofcontents				% Insert table of contents

\newpage

\section{Introduction}

The \emph{Dragonfly} project is an internal competence enhancement project for ÅF Technology employees. The goal is to combine technology, competence and experience from various engineering fields in order to construct a highly advanced quadrotor UAV system. The focus of the Flight Control Board development deals with low-level maneuvering of the aircraft, calculating motor command based on a feedback control system. Some of the major technologies deployed to attain this are control theory, electronics and software development.

\section{Use cases}


\usecase{uc:navigate}{UC:Navigate}
\scenario{sc:cxn.sunny}{SC:Nav-Sunny} GPS and map data available.
\requirement{req:nav.sunny.1}{Nav.Sunny.1} Navigate to a destination.
\requirement{req:nav.sunny.2}{Nav.Sunny.2} Navigate to start position. The start position is position of lift-off.

\scenario{sc:cxn.rainy}{SC:Nav-Rainy} GPS not available.
\requirement{req:nav.rainy.1}{Nav.Rainy.1} User shall be notified that navigation is unavailable with cause.

\usecase{uc:cxn}{UC:Connection}

\scenario{sc:cxn-ok}{SC:Cxn-Sunny}
\requirement{req:cxn.sunny.wifi}{Cxn.Sunny.1}: The device should have a functional WiFi connection.
\requirement{req:can.sunny.wless}{Cxn.Sunny.2}: The device should have a functional wireless connection.

\usecase{uc:cxn}{UC:Safety}

\subsection{Use case 3: Safety}

\begin{enumerate}
\labitem{req:safe.1}{Safety.1}: The device must have a panic-mode, where:
\begin{itemize}
\item engines are switched off.
\item last known position of the device is made available remotely.
\end{itemize}
\labitem{req:safe.2}{Safety.2}: The user must be able to set the DragonFly in panic-mode through the controller.
\labitem{req:safe.3}{Safety.3}: The device must avoid collisions.
\labitem{req:safe.4}{Safety.4}: In case of a collision, the device must automatically switch to panic-mode.
\end{enumerate}


\subsection{Use case 4: Control Flight}

\begin{enumerate}
\labitem{req:cf.1}{Control Flight.1}: The user must be able to control the Dragonfly manually through a RC transmitter.
\labitem{req:cf.2}{Control Flight.2}: The user must be able to control the Dragonfly manually through a mobile app.
\labitem{req:cf.3}{Control Flight.3}: The user must be able to switch between controlling through mobile app or RC and vice verse while flying.
\labitem{req:cf.4}{Control Flight.4}: The manual control through the RC must have the highest priority (over mobile app and navigation control).
\labitem{req:cf.5}{Control Flight.5}: The Dragonfly must have a self stabilizing control algorithm.
\labitem{req:cf.6}{Control Flight.6}: The user must be able to control the Dragonfly flying:
\begin{itemize}
\item Forwards.
\item Backwards.
\item Left.
\item Right.
\item Rotate.
\item Up.
\item Down.
\end{itemize}
\labitem{req:cf.7}{Control Flight.7}: When the user releases the control the Dragonfly must return to a stand still position.
\labitem{req:cf.8}{Control Flight.8}: The Dragonfly must be able to land on its own.
\labitem{req:cf.9}{Control Flight.9}: The Dragonfly must be able to fly in winds up to 10 m/s.
\end{enumerate}


\section{Example}
\subsection{It's still possible to use the vanilla sections}
And they will appear like use cases etc in the table of contents.
\usecase{uc:cross-reference}{UC:display-reference} Use case text. Use case text.Use case text.Use case text.Use case text.
\scenario{sc:cross-reference}{SC:display-reference} Scenario text. Scenario text. Scenario text. Scenario text. Scenario text. Scenario text.
\requirement{req:cross-reference}{REQ:display-reference} Requirement text. Requirement text. Requirement text. Requirement text. Requirement text. Requirement text.


\section{Example: Referring to requirements}
\subsection{Referring to reqs by itemlabel}
Use \verb!\ref!to refer to the \verb!\labitem! requirements. Let's consider requirement \ref{req:cf.5} and \ref{req:cf.6}.

\subsection{Referring to newcommand uc, scenario and requirements.}
It is preferred to use the \verb!\nameref! to refer to the use cases, scenarios and requirements.
Thus, we look at use case \nameref{uc:navigate} scenario \nameref{sc:cxn.sunny} and requirements \nameref{req:cxn.sunny.wifi}.

\subsection{Sandbox of referring to newcommands}
\nameref{uc:cross-reference}, scenario \nameref{sc:cross-reference} and reqirement \nameref{req:cross-reference}.


\begin{thebibliography}{99}
\bibitem{stenberg} Model-based Design Development and Control of a Wind Resistant Multirotor UAV, C. Månsson, D. Stenberg, Lunds Tekniska Högskola 2014
\end{thebibliography}

\end{document}                  % End of document
